\documentclass[11pt]{article}

\usepackage[ngerman]{babel}
\usepackage[utf8]{inputenc}
\usepackage[T1]{fontenc}
\usepackage{graphicx}
\usepackage{microtype}
\usepackage{amssymb,amsthm}
\usepackage[leqno]{amsmath}
\usepackage{units}

\usepackage{tikz}
\usetikzlibrary{decorations,shapes,arrows,scopes,fadings,plotmarks,calc,intersections,patterns,positioning
}
\usetikzlibrary{decorations.pathreplacing}
\usetikzlibrary{decorations.pathmorphing}
\usetikzlibrary{decorations.shapes}
\usetikzlibrary{decorations.markings}

\usepackage{pgfplots}
\usepackage{tikz-3dplot}

\usepackage[pdftex,active,tightpage]{preview}
\PreviewEnvironment{tikzpicture}
\setlength\PreviewBorder{1mm}

\begin{document}

\begin{tikzpicture}[scale=0.7]
	\def\boundary{(0,0) coordinate (dA) to [out=90,in=-30] +(-2,4) to [out=35,in=-175] +(6,2) to [out=-30,in=100] +(2,-4) to [out=-150,in=10] (dA)--cycle}
	\draw[very thick, line join=round
	, name path = bd] \boundary;

	\path[name path = arrow] (0,4)--(1,6.3)coordinate(ldA);
	\draw[-,thick,name intersections= {of = bd and arrow, by=x}] (x)--(ldA) node[above] {$ \partial A $ };
	\node at (-1,2) {$ A $ };
	\draw[-,thick] (2.9,1.3)--+(1.5,-1) node[right] {$ \tilde A $ };

	\clip \boundary;
	\begin{scope}[rotate=19]
		\draw[step=0.57,very thin] (-1,-1) grid (7,5);
	\end{scope}
\end{tikzpicture}

\end{document}
