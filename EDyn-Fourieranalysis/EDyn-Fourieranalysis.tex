\documentclass[paper=a4, fontsize=11.0pt, abstractoff, DIV12]{scrartcl}
\usepackage[utf8]{inputenc}
\usepackage[ngerman]{babel} % deutsche Rechtschreibung

\usepackage{graphicx} % Grafiken einbinden
\usepackage{amsmath} % AMS! Wichtig!
\usepackage{amsfonts} % mehr AMS
\usepackage{amssymb} % noch viel mehr
\usepackage[amssymb]{SIunits} % Einheiten anständig setzen
\usepackage{bbm} % fetter gedrucktes verfügbar machen, z.B. \mathbbm{1} = Eins mit Doppelstrich
\usepackage{nicefrac}
\usepackage[square, Euler, newVec]{simpleMath}
\usepackage{tikz}
\usepackage{marvosym}
\usepackage{hyperref}
\hypersetup{colorlinks,
            breaklinks=true,
            linkcolor=black,
            urlcolor=black,
            citecolor=black,
            bookmarksnumbered,
            pdfauthor={Alexander Eberspächer},
            pdftitle={Übungen Theoretische Physik}}

\title{Schnellkurs Fourieranalysis}
\author{Alexander Eberspächer}
\date{\today}

\begin{document}
\maketitle
\begin{abstract}
Im Schnelldurchlauf wollen wir die Fourieranalysis angehen. Neben Fourierreihen
sollen die diskrete Fouriertransformation sowie die kontinuierliche
Fouriertransformation kurz vorgestellt werden.
\\[0.5ex]
Literatur/Quelle: kondensiert aus Arfken\&Weber \cite{Arfken}.
\end{abstract}

\section{Fourierreihen}

\subsection{Definition}

Periodische Funktionen $f(x)$ (vgl. Wellen und harmonnische Schwingungen)
lassen sich in eine sogenannte Fourierreihe entwicklen:

\begin{equation}
f(x) = \frac{a_0}{2}+\sum\limits_{n=1}^{\infty}a_n\cos(nx) + \sum\limits_{n=1}^{\infty}b_n\sin(nx)\,,
\label{eq:Fourierreihe}
\end{equation}
wobei die Entwicklungskoeffizienten $a_n$ und $b_n$ durch
\begin{align}
a_n &= \frac{1}{\pi}\int\limits_{0}^{2\pi}f(x)\cos(nx)\,\dd x\\
b_n &= \frac{1}{\pi}\int\limits_{0}^{2\pi}f(x)\sin(nx)\,\dd x
\end{align}
mit $n=0,1,2,\dots$ gegeben sind. Diese Reihen konvergieren für auf $[0;2\pi]$
periodische Funktionen $f(x)$ gegen $f(x)$ falls $f(x)$ ``gutmütig'' ist
(hier: endliche Anzahl von Sprungstellen und Extremwerten).

\subsection{Idee hinter Fourierreihen}

Eine Schwingung $f(x)$ wird ein Bestandteile diskreter Frequenz zerlegt. In der
Akustik kennen Sie das als Obertöne. Die Entwickluing funktioniert, weil
die Sinus- und Kosinusfunktionen orthogonal sind:

\begin{align}
\int\limits_{0}^{2\pi} \sin(nx)\cos(mx)\,\dd x&=0\,,\\
\int\limits_{0}^{2\pi} \sin(nx)\sin(mx)\,\dd x &= \left\{\begin{array}{lr}\pi\delta_{mn}&\qquad m\ne 0\\0&\qquad m=0\end{array}\right.\,,\\
\int\limits_{0}^{2\pi} \cos(nx)\cos(mx)\,\dd x&= \left\{\begin{array}{lr}\pi\delta_{mn}&\qquad m\ne 0\\2\pi&\qquad m=n=0\end{array}\right.\,,\\
\end{align}

Diese Funktionen sind Eigenfunktionen zum Sturm-Liouville-Problem
\begin{equation*}
y'' + n^2 y = 0\,.
\end{equation*}

Die Entwicklung ist damit wie schon die Entwicklung nach Legendre-Polynomen
eine Entwickung nach \emph{orthogonalen Funktionen}. Die Koeffizienten
$\left\{\begin{array}{c}a_n\\b_n\end{array}\right\}$ sind damit die
Projektionen von $f(x)$ auf die Basisfunktionen
$\left\{\begin{array}{c}\cos(nx)\\\sin(nx)\end{array}\right\}$.

\subsection{Komplexe Darstellung}

Aus der reellen Fourierreihe \eqref{eq:Fourierreihe} ergibt sich mit Hilfe der
Eulerformel $\eto{\ii nx} = \cos(nx)+\ii\sin(nx)$ durch Ersetzen von $\sin(nx)$
und $\cos(nx)$ nach etwas Umsortieren die komplexe Darstellung
\begin{equation}
f(x) = \sum\limits_{n=-\infty}^{+\infty}c_n\eto{\ii nx}\,,
\end{equation}
wobei für $n> 0$
\begin{align}
c_n &= \frac{1}{2}(a_n - \ii b_n)\\
c_{-n} &= \frac{1}{2}(a_n + \ii b_n)
\end{align}
gilt. Für $n=0$ ist
\begin{equation}
c_0 = \frac{1}{2}a_0\,.
\end{equation}
Die Formel für $c_n$ ergibt sich unter der Verwenug der Linearität des
Integrals und der Eulerformel explizit zu
\begin{align}
c_n &= \frac{1}{2\pi}\left[ \int\limits_{0}^{2\pi} f(x)\left(\cos(nx) - \ii \sin(nx)\right)\,\dd x\right]\nonumber\\
&=\frac{1}{2\pi}\int\limits_{0}^{2\pi} f(x) \eto{-\ii nx}\,\dd x\, .
\label{eq:cn}
\end{align}

\Stopsign\hspace{1ex}In der komplexen Darstellung müssen wir für alle
Projektionen ein komplexes Skalarprodukt verwenden! Hier brauchen wir
komplexe Konjugation -- das Skalarprodukt der Basisfunktionen, die mit $n$
beziehungsweise $m$ indiziert werden ist dann\footnote{TODO: Physiker-Konvention...}
\begin{align}
\left<n,m\right> &= \int\limits_{0}^{2\pi}\left(\eto{\ii nx}\right)^*\eto{\ii mx} \,\dd x\\
& \left( =  \int\limits_{0}^{2\pi} \eto{\ii (m-n)x}\,\dd x = 2\pi \delta_{mn}\right)\,.\nonumber
\end{align}
Die Projektion $c_n = \left<n,f\right>$ von $f$ auf die Basisfunktion
$\eto{\ii n x}$ benötigt dann die komplexe Konjugation der Basisfunktion,
was das Minuszeichen im Exponenten von \eqref{eq:cn} im Rahmen einer
\glqq geometrischen\grqq Interpretation (Projektion!) erklärt.

\subsection{Funktionen mit anderer Periodizität}

Bisher haben wir die Periode $2\pi$ für unsere Funktion $f$ angenommen. Auch
möglich ist jede andere Periode; beispielweise die Periode $2L$. In diesem Fall
lautet die Fourierreihe
\begin{align}
f(x) = \frac{a_0}{2} + \sum\limits_{n=1}^{\infty}\left[a_n \cos\left(\frac{n\pi}{L}x\right) + b_n \sin\left(\frac{n\pi}{L}x\right)\right]\,,
\end{align}
die Entwicklungskoeffizienten sind in diesem Fall
\begin{align}
a_n &= \frac{1}{L}\int\limits_{-L}^{+L} f(t) \cos\left(\frac{n\pi}{L}x\right)\,\dd t\quad&n=0,1,2,\dots\\
b_n &= \frac{1}{L}\int\limits_{-L}^{+L} f(t) \sin\left(\frac{n\pi}{L}x\right)\,\dd t\quad&n=1,2,\dots
\end{align}
oder in komplexer Darstellung
\begin{align}
c_n = \frac{1}{2L}\int\limits_{-L}^{+L} f(t) \eto{-\ii n x}\,\dd t\qquad&n=0,1,2,\dots\\
\end{align}

Das Intervall auf, dem $f$ periodisch ist muss dabei nicht wie $[-L;+L]$
symmetrisch sein, jedes Intervall $[x_0; x_0+2L]$ ist in Ordnung.

\subsection{Beispiel}

TODO

\section{Diskrete Fouriertransformation}

Messdaten oder vom Computer erzeugte Daten liegen in der Regel nicht als
kontinuierliche Daten, sondern sind stattdessen als diskrete Punktmenge
gegeben. Für diese Fall wollen wir die sogenannte \emph{diskrete
Fouriertransformation} (DFT) einführen.

\subsection{Orthogonalität im diskreten Fall}

Für diskrete Daten müssen wir unseren Begriff von Orthogonalität neu fassen.
Aus den Integralen in den Skalarprodukten müssen Summen werden.

Wir betrachten Daten $f(t_k)$, die im Intervall $[0;T]$ für die $2N$ Zeiten
\begin{align}
t_k &= 0, \frac{T}{2N}, \frac{2T}{2N},\dots,\frac{(2N-1)T}{2N}\nonumber\\
&:=\frac{T}{2N}k\qquad\quad k=0,\dots,2N-1
\end{align}
aufgenommen sind. Wir wollen nun die Orthogonalität von
\begin{align*}
\eto{2\pi \ii p t_k/T}\\
\intertext{und}
\eto{2\pi \ii q t_k/T}
\end{align*}
zeigen, wobei $p$ und $q$ ganze Zahlen sein sollen. Wir ersetzen im Skalarprodukt
das Integral durch eine Summe und erhalten mit $s:=q-p$
\begin{align}
\sum\limits_{k=0}^{2N-1}\left[\exp{\left(\frac{2\pi\ii pt_k}{T}\right)} \right]^* \exp{\left(\frac{2\pi\ii pt_k}{T}\right)} &= \sum\limits_{k=0}^{2N-1}\exp{\left(\frac{2\pi\ii st_k}{T}\right)}\\
&=\sum\limits_{k=0}^{2N-1}\exp{\left(\frac{2\pi\ii sk}{2N}\right)}\,,
\end{align}
wobei im letzten Schritt die Definition von $t_k$ eingesetzt wurde. Dieser
Ausdruck ist eine endliche geometrische Reihe mit $2N$ Gliedern, deren
Partialsumme $s_{2N}$ sich mit $r=\eto{2\pi\ii s}$ zu
\begin{equation}
s_{2N} = \left\{ \begin{array}{lr}\frac{1-r^{2N}}{1r}  & \quad r\ne 1\quad (s\ne 0 \leftrightarrow p \ne q)\\2N &\quad r=1\quad(s = 0 \leftrightarrow p = q)\end{array}\right.
\end{equation}
berechnet\footnote{TODO}. Damit lautet die Orthogonalitätsrelation für die
Basisfunktionen $p$ und $q$ im diskreten Fall
\begin{align}
\left<p,q\right>=\sum\limits_{k=0}^{2N-1} \left[\exp\left(\frac{2\pi\ii pt_k}{T}\right)\right]^*\exp\left(\frac{2\pi\ii qt_k}{T}\right) = 2N\delta_{p,q\pm 2nN}
\label{eq:diskOrth}
\end{align}
mit ganzzahligen $p,q,n$.

\subsection{Die DFT}

wir definieren zunächst die Frequenzen
\begin{equation}
\omega_p = \frac{2\pi p }{T}\qquad p=0,1,\dots,2N-1\,.
\end{equation}
Die Exponentialterme sind damit also
\begin{equation*}
\eto{\pm 2\pi\ii pt_k} = \eto{\pm \omega_p t_k}\,.
\end{equation*}
Die Orthogonalitätsrelation schreibt sich dann als
\begin{equation}
\sum\limits_{k=0}^{2N-1} \left(\eto{\ii \omega_p t_k}\right)^* \eto{\ii \omega_q t_k} = 2N\delta_{q,p+2Nn}\,.
\end{equation}
Für eine Funktion $f(t_k)$ definieren wir nun
\begin{equation}
F(\omega_p) = \frac{1}{2N}\sum\limits_{k=0}^{2N-1}f(t_k)\eto{\ii\omega_pt_k}
\label{eq:DFT}
\end{equation}
als die sogenannte \emph{diskrete Fouriertransformation} von $f(t_k)$.

\subsubsection{Inverse DFT}

Um die inverse Transformation abzuleiten, multiplizieren wir zunächst \eqref{eq:DFT}
mit $\left(\eto{\ii\omega_p t_q}\right)^*$ und summieren über $p$. Es ergibt sich
\begin{align}
\sum\limits_{p=0}^{2N-1}F(\omega_p)\left(\eto{\ii\omega_p t_q}\right)^* &= \frac{1}{2N} \sum\limits_{p=0}^{2N-1} \left(\eto{\ii\omega_p t_q}\right)^* \sum\limits_{k=0}^{2N-1}f(t_k) \eto{\ii\omega_p t_q}\nonumber\\
&=\sum\limits_{k=0}^{2N-1}f(t_k) \underbrace{\frac{1}{2N}\sum\limits_{p=0}^{2N-1}\left(\eto{\ii\omega_p t_q}\right)^*\eto{\ii\omega_p t_k} }_{=\delta_{qk}}\,,
\end{align}
was wie angedeuet mit der Orthogonalitätsrelation \eqref{eq:diskOrth} die inverse Transformation
\begin{align}
\sum\limits_{p=0}^{2N-1}F(\omega_p)\eto{-\ii\omega_p t_q} &= f(t_k)
\end{align}
ergibt.

Beide Richtungen dieser Transformation sind in allen Computer-Bibliotheken
und Routinen zur \emph{schnellen Fouriertransformation} (FFT) implementiert.

\subsubsection{Interpretation und Parseval-Theorem}

Die Werte $F(\omega_p)$ interpretieren sich als Beitrag  von
$\eto{\ii\omega_p t_k}$ zum Signal $f(t_k)$. Weiterhin gilt das
Parseval-Theorem:
\begin{align}
\frac{1}{2N}\su{k=0}{2N-1}\abs{f(t_k)}^2 = \su{k=0}{2N-1}\abs{F(\omega_k)}^2\,.
\end{align}
Die linke Seite steht für die summierte Energie im ``Zeitbereich'', die
rechte Seite hingegen steht für die gesamte Energie im ``Frequenzbereich''.
Die Parseval-Relation drückt also aus, dass die Energie unabhängig von der
gewählten Darstellung (Zeit/Frequenz) ist. Mathematisch spiegelt sich das in
Unitarität der DFT -- die Länge eines Vektors hängt nicht von der gewählten
Basis ab.

\section{Von Fourierreihen zur Fouriertransformation}

Für auf $[-L;+L]$ periodische Funktionen haben wir Fourierreihen
kennengelernt, dort waren die Koeffizienten vor den Kosinus-/Sinus-Basisfunktionen
\begin{align*}
a_n &= \frac{1}{L}\inte{-L}{+L} f(t) \cos\left(\frac{n\pi t}{L}\right)\,\dd t\\
b_n &= \frac{1}{L}\inte{-L}{+L} f(t) \sin\left(\frac{n\pi t}{L}\right)\,\dd t\,.
\end{align*}

Damit können wir den den Übergang zu Funktionen mit ``unendlicher Periode''
vorbereiten. Wir setzen zunächst die Koeffizienten in die Fourierreihe ein
und erhalten
\begin{align}
f(x) &= \underbrace{\frac{1}{2L} \inte{-L}{+L}f(t)\,\dd t}_{a_0} + \frac{1}{L}\su{n=1}{+\infty}\cos\left(\frac{n\pi x}{L}\right)\underbrace{\inte{-L}{+L}f(t)\cos\left(\frac{n \pi x}{L}\right)\,\dd t}_{a_n} \nonumber\\ &\quad + \frac{1}{L}\su{n=1}{+\infty}\sin\left(\frac{n\pi x}{L}\right)\underbrace{\inte{-L}{+L}f(t)\sin\left(\frac{n \pi x}{L}\right)\,\dd t}_{b_n}\,.
\end{align}

Nutzt man hier die Linearität des Integrals sowie die trigonometrischen
Relationen
\begin{align*}
bla
\end{align*}


\section{Integraltransformationen und die Fouriertransformation als Spezialfall}

\subsection{Spezielle Eigenschaften der Fouriertransformation}

\nocite{*}

\bibliographystyle{ieeetr}

\bibliography{EDyn-Fourieranalysis}


\end{document}
